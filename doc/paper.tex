\documentclass[12]{article}

\usepackage[linkcolor=blue]{hyperref}
\usepackage{tabularx}
\usepackage{amsmath,amssymb,amsthm}

% Space and indenting
\usepackage[margin=1in]{geometry}
\usepackage{setspace}
\onehalfspacing

\begin{document}

\section{Introduction}

Social Networks websites such as Twitter or Facebook have been center of attention for many researchers/disciplines. In the last years there have been several attempts to create a worthy use of the data generated by these systems.

Applications on marketing, opinion polls, etc. have been developed from both industry and academia. Some notable examples are
\marginpar{Indiana, CMU, etc.}

Political networks have been studied as well. Whereas models of contagion/diffusion of political ideas/postures, or simply network structure,... Nevertheless, there is no study that shows how much (if any) do political networks resemble real life networks; furthermore, we don't know whether US political parties (for example) behave differently in these social networks.

This paper analyzes the structure and dynamics of US politicians On line social networks.

\section{Literature}

On XXX Lada observed the differences between blogs classified between Democrats and Republicans. They found that interestingly democrats referral networks (what other blogs/websites they mention in their articles) is more open than republicans as republicans tend to closer ties. We should expect some resemble in online social networks.

\section{Model}

The question might be, what drives two politicians to be connected (besides political party). In political networks, a good proxy for connectedness is co-sponsorship. Paper XXX shows that co-sponsorship is highly correlated with social connections between politicians.

Political networks are interesting for several reasons. Besides of predicting co-sponsorship, it reveals dynamics in political careers as well connected politicians can achieve higher positions easy. The question is, what comes first.

While interesting question, we won't review it in this document. What we will explore are the following questions

\begin{itemize}
	\item Are Democratic/Republican networks different? If they do, how different are they?
	\item Do online social network political networks resemble something to the real political networks?
	\item Does the position on the network affects popularity of the politicians?
	\item Who does follows each party?, how much followers do they share?
	\item What does each group talks about?
\end{itemize}

Using official data from Twitter, these are the questions that we will discuss in this paper. Were we find differences between parties, the whys will be discussed some other time.

\section{Data}

Data was collected using Twitter's API service. In particular, we recovered politicians' statuses updates (tweets) from the last 5 months, so we are mostly concerned on what do they talk about and not what is been said about them.

The set of politicians selected corresponds to the US House of Representatives and US Congress. Twitter accounts were extracted from each member's website (so we are including the official accounts only).

The data was extracted using R's package \href{http://github.com/gvegayon/twitterreport}{twitterreport}.

For the paper we identified three types of networks: Follower/Following (F/F) networks, conversation networks, and mention networks. In the F/F networks, two politicians are connected if they follow each other. While the most commonly analyzed type of graph, this is not of much interest since following an individual does not implies that the user actually receives information from him or interacts in any way with him. On the other hand, conversation networks are more meaningful, as these show actual interaction between individuals. In this context we say that two individuals are connected if they mention each other in a set of particular messages\footnote{In twitter, an individual can mention other user by just appending an ``at'' symbol (@) before the other individual's user name, for instance, if user abc wants to mention user xyz, it suffices to include the text @xyz in the message.}

\section{Networks}

\subsection{FF networks}

\subsection{Conversation networks}

\subsection{Mention networks}

\section{Who follows who}

\section{What are they saying}

\section{Can we measure some causality?}

\end{document}